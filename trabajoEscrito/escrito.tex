\documentclass[12pt,a4paper]{article}
\usepackage{graphicx}
\usepackage{ textcomp }
\setlength{\voffset}{-0.75in}
\setlength{\headsep}{5pt}

\begin{document}
	
	\begin{titlepage}
		\centering
		\includegraphics[width=0.4\textwidth]{./images/logo_ciencias}\par\
		{\LARGE\bf Facultad de Ciencias \par}
		{\LARGE\bf UNAM \par}
		\vspace{1cm}
		{\huge\bfseries Metodolog\'ia de desarrollo de software: Crystal Clear\par}
		\vspace{2cm}
		{\Large\itshape Oscar Eduardo Villa Chio\par}
		{\Large\itshape Jorge Luis Garc\'ia Flores\par}
		{\Large\itshape Enrique Antonio Bernal Cedillo\par}
		{\Large\itshape V\'ictor Zamora Guti\'errez\par}
		\vspace{0.5cm}
		{\large Riesgo Tecnol\'ogico\par 6\textdegree \ Semestre\par}
		% Bottom of the page
		{\large 22 de abril del 2016}
	\end{titlepage}
	
	\section{Introducci\'on}
	
	\subsection{Objetivo general}
	El objetivo general de este trabajo es introducir las metodolog\'ias {\itshape Crystal} y en particular la metodolog\'ia {\itshape Crystal Clear}.
	
	\subsection{Objetivos espec\'ificos}
	\begin{itemize}
		\item Mostrarle a los alumnos de la clase de riesgo tecnol\'ogico c\'omo trabajar con la metodolog\'ia {\itshape Crystal Clear}.
		
		\item Otorgar una herramienta para terminar trabajos de peque\~no alcance en tiempos razonables.
	\end{itemize}
	
	\subsection{Alcance del trabajo}
	Clase de Riesgo Tecnol\'ogico.
	\pagebreak
	\section{Contenido}
	\subsection*{Creador}
	Alistair Cockburn se dedica a las ciencias de la computaci\'on. Fue uno de los firmantes del {\itshape manifiesto de metodolog\'ias \'agiles}, un documento muy importante para la ingenier\'ia de software, que habla acerca de m\'etodos ligeros de desarrollo. Tambi\'en fue creador de su propia familia de metodolog\'ias llamada {\itshape Crystal} a mitades de los a\~nos 90's, las cuales han sido un gran aporte para la ingenier\'ia de software.
	\\
	En particular, hablaremos de la tecnolog\'ia {\itshape Crystal Clear}. En su libro, Alistair explica que dicha metodolog\'ia fue desarrollada para la elaboraci\'on de proyectos peque\~nos (de 8 integrantes como m\'aximo).
Para obtener esta metodolog\'ia, Alistair realiz\'o investigaci\'on sobre grupos de desarrollo peque\~nos e identific\'o patrones en los grupos exitosos.\\
	Los puntos principales que las metodolog\'ias {\itshape Crystal} toman en cuenta son los siguientes:\\
	\begin{itemize}		
		\item El aspecto humano del equipo

		\item El tama\~no del equipo o los componentes
	
		\item Comunicaci\'on directa entre componentes o integrantes
		
		\item Las pol\'iticas a seguir
	
		\item Que se cuente con un \'unico espacio f\'isico de trabajo para la tarea espec\'ifica a realizar.
	\end{itemize}

	\subsection*{Manifiesto \'agil}
	Del 11 al 13 de febrero del  2001 se llev\'o acabo una reuni\'on organizada por Kent Beck. Se reunieron 17 especialistas en modelos de desarrollo de software, quienes crearon el t\'ermino {\itshape Metodolog\'ias \'agiles} como alternativa a las metodolog\'ias formales que existian en la \'epoca (por ejemplo {\itshape SCRUM} y {\itshape DSDM}).
	Los involucrados en esta reuni\'on fueron conocidos como firmantes del manifiesto \'agil. Ellos fueron:
	\begin{itemize}	
		\item- Kent Beck
		\item- Alistair Cockburn
		\item- Mike Beedle
		\item- Arie van Bennekum
		\item- Ward Cunningham
		\item- James Grenning
		\item- Martin Fowler
		\item- Jim Highsmith
		\item- Andrew Hunt
		\item- Ron Jeffries
		\item- Jon Kern
		\item- Brian Marick
		\item- Robert Cecil Martin
		\item- Steve Mellor
		\item- Ken Schwaber
		\item- Jeff Sutherland
		\item- Dave Thomas
		\\
	\end{itemize}		

	\subsection*{Metodolog\'ias \'Agiles}
	La familia {\itshape Crystal} forma parte de la categor\'ia de Metodolog\'ias \'Agiles de desarrollo de software, las cuales tienen como principales caracter\'isticas:
	\begin{itemize}
		\item Ser una alternativa a las metodolog\'ias tradicionales de desarollo de software.
		
		\item Valorar a las personas e interacciones por encima de las herramientas, procesos y productos.
		
		\item Que el software, adem\'as de funcionar, cuente con una documentaci\'on comprensiva.
		
		\item Tener un n\'umero bajo de intermediarios en el proyecto, y mantener la comunicaci\'on con el cliente durante el desarrollo.
		
		\item Utilizar poco tiempo en la toma de decisiones, para evitar p\'erdidas importantes de tiempo.
		
		\item Tener una respuesta \'agil ante cualquier cambio, ya sea en requerimientos, herramientas de desarollo, el equipo u otras cuestiones.
		\\
	\end{itemize}
		
	\subsection*{Metodolog\'ias Crystal}
	Alistair desarroll\'o diferentes metodolog\'ias basandose en los equipos de trabajo que requer\'ian diferentes estrategias para resolver diferentes problemas.
	Para esto Alistair us\'o una distinci\'on de colores para cada miembro de la familia crystal para indicar el "peso". Estas metodolog\'ias son:
	\begin{itemize}
		\item Crystal Clear  (1-6)
		\item Crystal Yellow (7-20)
		\item Crystal Orange (21-40)
		\item Crystal Orange Web (21-40)
		\item Crytal Red (41-80)
		\item Crystal Maroon (81-200)
		\item Crystal Diamond (201-500)
		\item Crystal Sapphire (800-superior)
	\end{itemize}
	  
	En donde se indica el n\'umero de integrantes que debe tener el equipo de trabajo.\\
\\
	Las metodolog\'ias {\itshape Crystal} tienen un enfoque importante en:
	\begin{itemize}
		\item Gente 
		\item Interacci\'on
		\item Comunidad
		\item Habilidades
		\item Talentos
		\item Comunicaci\'on
	\end{itemize}

\subsection*{Roles en la familia {\itshape Crystal}}
	
	
	La familia {\itshape Crystal} define roles para mejorar el rendimiento en el desarrollo del software, los cuales son:
	\begin{itemize}
		\item Patrocinador:\\
		Es el encargado de asignar los fondos necesarios para el proyecto, adem\'as de mantener una visi\'on a largo plazo respecto a este. Tiene la capacidad de tomar la decicisi\'on de detener el proceso de desarrollo.
		\item Usuario Experto:\\
		Es el encargado de hacer una lista en la cu\'al se indican los objetivos a realizar, el personal requerido para lograrlo, los casos de uso y los requerimientos del sistema. 
		\item Dise\~nador:\\
		Es la persona encargada de hacer los dise\~nos de los sistemas principales, adem\'as de mantener al equipo de trabajo en el mismo canal de comunicaci\'on para realizar un desarrollo homog\'eneo.
		\item Dise\~nador-Programador:\\
		Es el encargado (junto con el dise\~nador) de producir las plantillas o borradores de las pantallas del software.
		\item Experto en negocios:\\
		Es el encargado de analizar estrategias a usar. Es quien define qu\'e pol\'iticas ser\'an fijas en el desarollo del software.
		\item Coordinador:\\
		Es el encargado de mostrar la estructura y el estado del proyecto a los patrocinadores . Se encarga de reducir los conflictos.
		\item Verificador:\\
		Realiza las pruebas necesarias para verificar la funcionalidad del sistema y se encarga de elaborar reportes.
		\item Escritor:\\
	Realiza los manuales de usuario. Esta persona debe tener un vocabulario amplio y una gran fluidez para la expresi\'on de ideas. Debe ser capaz de especificar claramente el c\'omo se debe usar el software dado.      
	\end{itemize}

	\subsection*{Desarrollo de la metodolog\'ia Crystal Clear}
		
	Fue de esta manera que se encontr\'o que varios equipos realizaban las mismas acciones:	
	\begin{itemize}
	\item Sentar a la gente cerca. Que se comuniquen seguido y en un ambiente de amabilidad y de buena gana.
	
	\item Alejar los temas burocr\'aticos lo m\'as posible de los desarrolladores.
	
 	\item Dejar que los desarrolladores dise\~nen el sistema y tomen las decisiones que consideren apropiadas.
	
	\item Tener a un usuario involucrado en el proyecto. Esto ayuda a encontrar errores, realizar pruebas y perfeccionar el dise\~no del sistema.
	
	\item Tener un conjunto de pruebas de regresi\'on automatizadas para asegurarse de que el software funciona correctamente en cada paso del desarrollo.
	
	\item Producir funcionalidad entregable desde temprano y de manera frecuente.
	\\
	\end{itemize}
					
	La metodolog\'ia Crystal Clear se puede describir como sigue:\\
	
	El dise\~nador principal junto con otras 2 a 7 personas trabajan en un espacio peque\~no como una habitaci\'on (o varias habitaciones adyacentes) con muestras visuales de informaci\'on (como pizarrones) sin ninguna distracci\'on y con acceso directo a usuarios clave.\\
	
	Deben entregar c\'odigo usable y probado aproximadamente cada 3 meses (puede ser menos dependiendo de la duraci\'on del proyecto). Esto para mantener la comunicaci\'on tanto con el equipo como con el cliente, la cual es esencial para no perder tiempo.\\
	
	Gracias a estas entregas frecuentes de prototipos, es posible mejorar por medio de retroalimentaci\'on con el cliente. Esta es otra de las caracter\'isticas importantes de la metodolog\'ia Crystal Clear.
	
	La metodolog\'ia Crystal Clear no pretende ser la mejor, sino tan s\'olo ser "suficiente". No asegura un producto perfecto, pero s\'i funcional y sin sacrificar el tiempo libre de los involucrados.\\
	
	\section{Conclusiones}
	Es una estrategia fiable para el desarrollo de productos de software ya que es f\'acil de enteder y es efectiva para un ambiente de trabajo tranquilo y eficaz. 
Promueve la comunicaci\'on cara a cara con cada miembro relacionado con el proyecto adem\'as de poseer una estructura jer\'arquica bien definida para cada rol de trabajo.\\
Para las empresas peque\~nas que quieren entregar productos de calidad decente sin acabarse su presupuesto, creemos que definitivamente esta es la mejor estrategia.
Sin embargo, para productos de calidad m\'as alta, se recomienda utilizar una estrategia m\'as rigurosa. 
	
	\section{Bibliografía}
	Cockburn, A., (2004) {\itshape Crystal Clear: A Human-Powered Methodology for Small Teams: A Human-Powered Methodology for Small Teams}. Boston, Addison-Wesley Professional
\\
\\Agile Alliance, (2015) "12 Principles Behind the Agile Manifesto" en {\itshape Agile Alliance.} [En l\'inea.] disponible en:
\\
https://www.agilealliance.org/agile101/12-principles-behind-the-agile-manifesto/
[Accesado el d\'ia 20 de abril de 2016]

\end{document}