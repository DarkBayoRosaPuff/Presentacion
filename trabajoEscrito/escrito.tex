\documentclass[12pt,a4paper]{article}
\usepackage{graphicx}
\usepackage{ textcomp }
\setlength{\voffset}{-0.75in}
\setlength{\headsep}{5pt}

\begin{document}
	
	\begin{titlepage}
		\centering
		\includegraphics[width=0.4\textwidth]{./images/logo_ciencias}\par\
		{\LARGE\bf Facultad de Ciencias \par}
		{\LARGE\bf UNAM \par}
		\vspace{1cm}
		{\huge\bfseries Metodolog\'ia de desarrollo de software: Crystal Clear\par}
		\vspace{2cm}
		{\Large\itshape Oscar Eduardo Villa Chio\par}
		{\Large\itshape Jorge Luis Garc\'ia Flores\par}
		{\Large\itshape Enrique Antonio Bernal Cedillo\par}
		{\Large\itshape V\'ictor Zamora Guti\'errez\par}
		\vspace{0.5cm}
		{\large Riesgo Tecnol\'ogico\par 6\textdegree \ Semestre\par}
		% Bottom of the page
		{\large 22 de abril del 2016}
	\end{titlepage}
	
	\section{Introducci\'on}
	
	\subsection{Objetivo general}
	El objetivo general de este trabajo es introducir la metodolog\'ia Crystal Clear y explicar sus usos a trav\'es de varios ejemplos.
	
	\subsection{Objetivos espec\'ificos}
	\begin{itemize}
		\item Mostrarle a los alumnos de la clase de riesgo tecnol\'ogico c\'omo trabajar con la metodolog\'ia Crystal Clear.
		
		\item Otorgar una herramienta para terminar trabajos de peque\~no alcance en tiempos razonables.
	\end{itemize}
	
	\subsection{Alcance del trabajo}
	Clase de Riesgo Tecnol\'ogico.
	\pagebreak
	
	\section{Contenido del trabajo}
	La metodologia Crystal Clear forma parte de una familia de metodologías llamada Crystal.
	
	\subsection*{Metodologias agiles}
	Dicha familia forma parte de la categoria de Metodologías agiles de desarrollo de software, las cuales tienen como principales características:
	\begin{itemize}
		\item Ser una alternativa a las metodologías tradicionales de desarollo de software.\\
		(Debido a su antiguedad.)
		
		\item Valorar a las personas e interacciones por encima de las herramientas, procesos y productos.
		
		\item Que el software aparte de funcionar cuente con una documentación comprensiva
		
		\item Reducir la cantidad de intermediarios en el proyecto, y mantener la comunicacion con el cliente durante el desarrollo
		
		\item Reducir el tiempo utilizado en la toma de decisiones, para evitar pérdidas importantes de tiempo.
		
		\item Respuesta agil ante cualquier cambio ya sea en requerimientos, herramientas de desarollo, el equipo y demas cuestiones.
		\\
	\end{itemize}
	[Tomado de:\\
	https://www.agilealliance.org/agile101/12-principles-behind-the-agile-manifesto/]\\
	
	\subsection*{Metodologias Crystal}	
	Tomando en cuensta esto, entonces los puntos principales que las metodologias Crystal toman en cuenta son los siguientes:\\
	\begin{itemize}		
		\item El aspecto humano del equipo

		\item El tamaño del equipo (o de componentes)		
	
		\item Comunicación directa entre componentes o integrantes
		
		\item Políticas a seguir distintas
	
		\item Contar con un unico espacio físico de trabajo, para la tarea específica a realizar.
	\end{itemize}
	
	\subsection*{Metodologia Crystal Clear}
	(Arreglar todas las fuentes y cosas que estan por todos lados :v como esta k aqui tenia shovimon)\\
	Crystal Clear A Human-Powered Methodology For Small Teams, including The Seven Properties of Effective Software Projects \linebreak
		
	La metodolog\'ia \textit{Crystal Clear} fue desarrollada por Alistair Cockburn.\\
	En su libro, Alistair explica que dicha metodolog\'ia fue desarrollada para la elaboraci\'on de proyectos peque\~nos (de un aproximado de 8 integrantes como máximo, dentro de un area en com\'un). \\
	
	Para obtener esta metodolog\'ia, Alistair realiz\'o investigaci\'on sobre grupos peque\~nos de desarrollo que cumplian con \'exito lo requerido.\\
	Fue de esta manera que se encontr\'o que varios equipos realizaban las mismas acciones:	
	\begin{itemize}
	\item Sentar a la gente cerca. Que se comuniquen seguido y en un ambiente de amabilidad y de buena gana.
	
	\item Alejar los temas burocr\'aticos lo m\'as posible de los desarrolladores.
	
 	\item Dejar que los desarrolladores dise\~nen el sistema y tomen las decisiones que consideren apropiadas.
	
	\item Tener a un usuario involucrado en el proyecto.\\
	(Este ayuda a encontrar errores, realizar pruebas y perfeccionar el dise\~no del sistema)
	
	\item Tener un conjunto de pruebas de regresi\'on automatizadas para asegurarse de que el software funciona correctamente en cada paso del desarrollo.
	
	\item Producir funcionalidad entregable desde temprano y seguido (?).
	\\
	\end{itemize}
	
	La metodolog\'ia Crystal Clear se puede describir como sigue:\\
	
	El dise\~nador principal junto con otras 2 a 7 personas trabajan en un espacio peque\~no como una habitaci\'on (o varias habitaciones adyacentes) con muestras visuales de informaci\'on (como pizarrones) sin ninguna distracci\'on y con acceso directo a usuarios clave.\\
	
	Deben entregar c\'odigo usable y probado aproximadamente cada 3 meses o menos a los clientes dependiendo de la duracion del proyecto.\\
	Por lo que la integracion de componentes del proyecto tambien debe realizarse continuamente.\\
	(Esto para mantener la comunicacion tanto con el equipo como con el cliente, la cual es esencial para no perder tiempo)\\
	
	Gracias a estas entregas frecuentes de prototipos, es posible mejorar por medio de retroalimentacion con el cliente, esta es otra de las caracteristicas importantes de la metodologia Crystal Clear.
	
	La metodolog\'ia Crystal Clear no pretende ser la mejor, sino tan s\'olo ser "suficiente" (No asegura un producto perfecto, pero sí funcional y sin sacrificar el tiempo de los involucrados).\\
	
	(Seria bueno incluir una comparacion entre crystal clear y las demas metodologias crystal, mas que nada las diferencias por las que se distinguen el cuando usar una en lugar de otra)\\
	
	(Identificar los roles y agregar mas informacion de esta pagina, que explica todo chido
	http://es.slideshare.net/fvalerolujano/metodologias-giles-crystal-clear)\\
	
	(Un "ejemplo concreto", o ejemplos (como dice la introduccion), o cambiar la introduccion)\\
	
	(Agregar muchas fuentes mas confiables y asi :V como quiere waifu, bien referenciadas)\\
	
	(Obviamente ampliar y mejorar la conclusion)\\
	
	\section{Conclusiones}
	Esta chida :v Es sencilla, rápida y ayuda a guiar el desarrollo, y darse una idea de lo que hay que hacer.\\
	Especialmete util para proyectos y trabajos pequeños, ya que muchas veces subestimamos la importancia de la comunicación directa, cara a cara o trabajar en un espacio en comun.\\
	
	\section{Bibliografía}
	
	\section{Anexos}
	(Solo dejar en caso de requerirlos)

\end{document}