\documentclass[12pt,a4paper]{article}
\usepackage{graphicx}
\usepackage{ textcomp }
\setlength{\voffset}{-0.75in}
\setlength{\headsep}{5pt}
\begin{document}
\begin{titlepage}
	\centering
	\includegraphics[width=0.5\textwidth]{./images/logo_ciencias}\par\
	{\scshape\LARGE Facultad de Ciencias \par}
	{\scshape\LARGE UNAM \par}
	\vspace{1cm}
	{\huge\bfseries Metodolog\'ia de desarrollo de software: Crystal Clear\par}
	\vspace{2cm}
	{\Large\itshape Oscar Eduardo Villa Chio\par}
	{\Large\itshape Jorge Luis Garc\'ia Flores\par}
	{\Large\itshape Enrique Antonio Bernal Cedillo\par}
	{\Large\itshape V\'ictor Zamora Guti\'errez\par}
	\vspace{0.5cm}
	{\large Riesgo Tecnol\'ogico\par 6\textdegree \ Semestre\par}
% Bottom of the page
	{\large 22 de abril del 2016}
\end{titlepage}
\section{Introducci\'on}
\subsection{Objetivo general}
El objetivo general de este trabajo es introducir la metodolog\'ia Crystal Clear y explicar sus usos a trav\'es de varios ejemplos.
\subsection{Objetivos espec\'ificos}
\subsection{Alcance del trabajo}
\end{document}