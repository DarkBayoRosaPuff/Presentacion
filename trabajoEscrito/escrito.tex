\documentclass[12pt,a4paper]{article}
\usepackage{graphicx}
\usepackage{ textcomp }
\setlength{\voffset}{-0.75in}
\setlength{\headsep}{5pt}

\begin{document}
	
	\begin{titlepage}
		\centering
		\includegraphics[width=0.4\textwidth]{./images/logo_ciencias}\par\
		{\LARGE\bf Facultad de Ciencias \par}
		{\LARGE\bf UNAM \par}
		\vspace{1cm}
		{\huge\bfseries Metodolog\'ia de desarrollo de software: Crystal Clear\par}
		\vspace{2cm}
		{\Large\itshape Oscar Eduardo Villa Chio\par}
		{\Large\itshape Jorge Luis Garc\'ia Flores\par}
		{\Large\itshape Enrique Antonio Bernal Cedillo\par}
		{\Large\itshape V\'ictor Zamora Guti\'errez\par}
		\vspace{0.5cm}
		{\large Riesgo Tecnol\'ogico\par 6\textdegree \ Semestre\par}
		% Bottom of the page
		{\large 22 de abril del 2016}
	\end{titlepage}
	
	\section{Introducci\'on}
	
	\subsection{Objetivo general}
	El objetivo general de este trabajo es introducir la metodolog\'ia Crystal Clear y explicar sus usos a trav\'es de varios ejemplos.
	
	\subsection{Objetivos espec\'ificos}
	\begin{itemize}
		\item Mostrarle a los alumnos de la clase de riesgo tecnol\'ogico c\'omo trabajar con la metodolog\'ia Crystal Clear.
		
		\item Otorgar una herramienta para terminar trabajos de peque\~no alcance en tiempos razonables.
	\end{itemize}
	
	\subsection{Alcance del trabajo}
	Clase de Riesgo Tecnol\'ogico.
	\pagebreak
	
	\section{Creador}
	Alistair Cockburn, es un comput\'ologo de ciencias de la computaci\'on,fue uno de los firmantes del manifiesto de metodologias \'agiles como tambien creador de su propia familia de metodolog\'ias llamada Cristal Clear a mitades de los a\~nos 90's las cuales han sido un gran aporte para la ingenieria de software.
	\\
	En su libro, Alistair explica que dicha metodolog\'ia fue desarrollada para la elaboraci\'on de proyectos peque\~nos (de un aproximado de 8 integrantes como máximo, dentro de un \'area en com\'un).
	Para obtener esta metodolog\'ia, Alistair realiz\'o investigaci\'on sobre grupos peque\~nos de desarrollo que cumplian con \'exito lo requerido.\\
	Tomando en cuensta esto, entonces los puntos principales que las metodologias Crystal toman en cuenta son los siguientes:\\
	\begin{itemize}		
		\item El aspecto humano del equipo

		\item El tamaño del equipo (o de componentes)		
	
		\item Comunicación directa entre componentes o integrantes
		
		\item Políticas a seguir distintas
	
		\item Contar con un unico espacio físico de trabajo, para la tarea específica a realizar.
	\end{itemize}

	\subsection{Manifiesto \'agil}
	Entre febrero 11-13 del a\~no 2001 se llevo acabo una reunion organizada por Kent beck en el cual reunio a 17 especialistas en modelos de desarrollo de software, en esta reuni\'on se creo el t\'ermino de Metodolog\'ias \'agiles como alternativa a las metodolog\'ias formales que existian en la \'epoca como SCRUM y DSDM.
	Los involucrados en esta reuni\'on fueron conocidos como firmantes del manifiesto \'agil, los cuales fueron:
	\begin{itemize}	
		\item- Kent Beck
		\item- Alistair Cockburn
		\item- Mike Beedle
		\item- Arie van Bennekum
		\item- Ward Cunningham
		\item- James Grenning
		\item- Martin Fowler
		\item- Jim Highsmith
		\item- Andrew Hunt
		\item- Ron Jeffries
		\item- Jon Kern
		\item- Brian Marick
		\item- Robert Cecil Martin
		\item- Steve Mellor
		\item- Ken Schwaber
		\item- Jeff Sutherland
		\item- Dave Thomas
		\\
	\end{itemize}		

	\subsection*{Metodologias agiles}
	Dicha familia forma parte de la categoria de Metodologías agiles de desarrollo de software, las cuales tienen como principales características:
	\begin{itemize}
		\item Ser una alternativa a las metodologías tradicionales de desarollo de software.\\
		(Debido a su antiguedad.)
		
		\item Valorar a las personas e interacciones por encima de las herramientas, procesos y productos.
		
		\item Que el software aparte de funcionar cuente con una documentación comprensiva
		
		\item Reducir la cantidad de intermediarios en el proyecto, y mantener la comunicacion con el cliente durante el desarrollo
		
		\item Reducir el tiempo utilizado en la toma de decisiones, para evitar pérdidas importantes de tiempo.
		
		\item Respuesta agil ante cualquier cambio ya sea en requerimientos, herramientas de desarollo, el equipo y demas cuestiones.
		\\
	\end{itemize}
	
	
	[Tomado de:\\
	https://www.agilealliance.org/agile101/12-principles-behind-the-agile-manifesto/]\\
	
	\subsection*{Desarrollo de la metodologia Crystal Clear}
		
	Fue de esta manera que se encontr\'o que varios equipos realizaban las mismas acciones:	
	\begin{itemize}
	\item Sentar a la gente cerca. Que se comuniquen seguido y en un ambiente de amabilidad y de buena gana.
	
	\item Alejar los temas burocr\'aticos lo m\'as posible de los desarrolladores.
	
 	\item Dejar que los desarrolladores dise\~nen el sistema y tomen las decisiones que consideren apropiadas.
	
	\item Tener a un usuario involucrado en el proyecto.\\
	(Este ayuda a encontrar errores, realizar pruebas y perfeccionar el dise\~no del sistema)
	
	\item Tener un conjunto de pruebas de regresi\'on automatizadas para asegurarse de que el software funciona correctamente en cada paso del desarrollo.
	
	\item Producir funcionalidad entregable desde temprano y seguido (?).
	\\
	\end{itemize}
	Con esto hiz\'o que las metodolog\'ias Crystal tuevieran un enfoque importante en:
	\begin{itemize}
		\item Gente 
		\item Interacci\'on
		\item Comunidad
		\item Habilidades
		\item Talentos
		\item Comunicaci\'on
	\end{itemize}
			
	\subsection*{Metodologias Crystal}		
	La metodolog\'ia Crystal Clear se puede describir como sigue:\\
	
	El dise\~nador principal junto con otras 2 a 7 personas trabajan en un espacio peque\~no como una habitaci\'on (o varias habitaciones adyacentes) con muestras visuales de informaci\'on (como pizarrones) sin ninguna distracci\'on y con acceso directo a usuarios clave.\\
	
	Deben entregar c\'odigo usable y probado aproximadamente cada 3 meses o menos a los clientes dependiendo de la duracion del proyecto.\\
	Por lo que la integracion de componentes del proyecto tambien debe realizarse continuamente.\\
	(Esto para mantener la comunicacion tanto con el equipo como con el cliente, la cual es esencial para no perder tiempo)\\
	
	Gracias a estas entregas frecuentes de prototipos, es posible mejorar por medio de retroalimentacion con el cliente, esta es otra de las caracteristicas importantes de la metodologia Crystal Clear.
	
	La metodolog\'ia Crystal Clear no pretende ser la mejor, sino tan s\'olo ser "suficiente" (No asegura un producto perfecto, pero sí funcional y sin sacrificar el tiempo de los involucrados).\\
	
	\subsection{Familia Crystal}
	Alistair desarroll\'o diferentes metodolog\'ias basandose en los equipos de trabajo que requerian diferentes estrat\'egias para resolver diferentes problemas.
	Para esto Alistair \'uso una distinci\'on de colores para cada miembro de la familia crystal para indicar el "peso", estos m\'etodos son:
	\begin{itemize}
		\item Crystal Clear  (1-6)
		\item Crystal Yellow (7-20)
		\item Crystal Orange (21-40)
		\item Crystal Orange Web (21-40)
		\item Crytal Red (41-80)
		\item Crystal Maroon (81-200)
		\item Crystal Diamond (201-500)
		\item Crystal Sapphire (800-superior)
	\end{itemize}
	  
	En donde se indica el n\'umero de integrantes que debe tener el equipo de trabajo basandose en la metodolog\'ia a usar.
	
	La familia Crystal define roles para as\'i mejorar el rendimiento en el desarrollo del software los cuales son:
	\begin{itemize}
		\item Pratocinador:\\
		Es el encargado en asignar los fondos necesarios par el proyecto, además de mantener una visi\'on a largo plazo para mantener las prioridades c\'omo tamb\'ien tener la facultada de tomar la decicisi\'on de detener el proceso desarrollo.
		\item Usuario Experto:\\
		Esta encargado de realizar y especificar una lista en el cu\'al indica los objetivos a realizar, tamb\'ien del personal requerido para lograrlo y el como definir los casos de uso y los requerimientos del sistema. 
		\item Diseñador:\\
		Es la persona encargada en hacer los diseños de los sistemas principales, además de mantener al equipo de trabaja en el mismo canal de comunicación para hacer un desarrollo homog\'eneo, ya que el diseñador cuenta con la experiencia necesaria para realizar estas tareas.
		\item Diseñador-Programador:\\
		Es el encargado (junto con el diseñador) de producir las plantillas o borradores de las pantallas a usar el software.
		\item Experto en negocios:\\
		Es el encargado en verificar que estrateg\'ias a usar o que pol\'iticas ser\'an fijas en el desarollo del software.
		\item Coordinador:\\
		Es el encargado de dar la estructura y del estado del proyecto a los patrocinadores además de que facilita las conversaciones para reducir los conflictos.
		\item Verificador:\\
		Realiza las pruebas necesarias para verificar la funcionaidad del sistema, además de estar c\'omo encargado de la realizaci\'on de reportes.
		\item Escritor:\\
		Realiza los manuales de usuario, esta persona debe tener un vocabulario amplio y una gran fluides para la expresi\'on de ideas para sintetizar y especificar claramente el c\'omo se debe usar el software dado.      
	\end{itemize}
	
	 
	
	(Identificar los roles y agregar mas informacion de estas paginas, que explican chido:\\
	http://es.slideshare.net/fvalerolujano/metodologias-giles-crystal-clear \\
	https://en.wikiversity.org/wiki/Crystal\_Methods \\
	http://ingenieriadesoftware.mex.tl/59189\_Metodologia-Crystal.html \\
	)\\
	
	(Un "ejemplo concreto", o ejemplos (como dice la introduccion), o cambiar la introduccion)\\
	
	(Agregar muchas fuentes mas confiables y asi :V como quiere waifu, bien referenciadas)\\
	
	(Obviamente ampliar y mejorar la conclusion)\\
	
	\section{Conclusiones}
	Es una estrateg\'ia fiable para el desarrollo de productos de software ya que es fac\'il de enteder, es efectiva para un ambiente de trabajo tranquilo y eficaz ya que promueve la comunicaci\'on cara a cara con cada miembro relacionado con el proyecto además de poseer una estructura jerarqu\'ica bien definida para cada role de trabajo que competen a una interacci\'on constante entre s\'i para evitar impactos tanto en calidad c\'omo de tiempo.
	  
	
	\section{Bibliografía}
	
	\section{Anexos}
	(Solo dejar en caso de requerirlos)

\end{document}