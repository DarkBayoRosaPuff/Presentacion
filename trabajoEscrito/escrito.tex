\documentclass[12pt,a4paper]{article}
\usepackage{graphicx}
\usepackage{ textcomp }
\setlength{\voffset}{-0.75in}
\setlength{\headsep}{5pt}
\begin{document}
\begin{titlepage}
	\centering
	\includegraphics[width=0.5\textwidth]{./images/logo_ciencias}\par\
	{\LARGE\bf Facultad de Ciencias \par}
	{\LARGE\bf UNAM \par}
	\vspace{1cm}
	{\huge\bfseries Metodolog\'ia de desarrollo de software: Crystal Clear\par}
	\vspace{2cm}
	{\Large\itshape Oscar Eduardo Villa Chio\par}
	{\Large\itshape Jorge Luis Garc\'ia Flores\par}
	{\Large\itshape Enrique Antonio Bernal Cedillo\par}
	{\Large\itshape V\'ictor Zamora Guti\'errez\par}
	\vspace{0.5cm}
	{\large Riesgo Tecnol\'ogico\par 6\textdegree \ Semestre\par}
% Bottom of the page
	{\large 22 de abril del 2016}
\end{titlepage}
\section{Introducci\'on}
\subsection{Objetivo general}
El objetivo general de este trabajo es introducir la metodolog\'ia Crystal Clear y explicar sus usos a trav\'es de varios ejemplos.
\subsection{Objetivos espec\'ificos}
\subsection{Alcance del trabajo}
\pagebreak
\section{Contenido del trabajo}
Crystal Clear
A Human-Powered Methodology For Small Teams,
including
The Seven Properties of Effective Software Projects
\linebreak
La metodolog\'ia \textit{Crystal Clear} fue desarrollada por Alistair Cockburn.
En su libro, Alistair explica que dicha metodolog\'ia fue desarrollada para la elaboraci\'on de proyectos peque\~nos.
Para obtener esta metodolog\'ia, Alistair realiz\'o investigaci\'on sobre grupos peque\~nos de desarrollo que tuvieron \'exito. Se encontr\'o que varios equipos realizaban las mismas acciones:
Sentar a la gente cerca. Que se comuniquen seguido y en un ambiente de amabilidad y de buena gana.
Alejar los temas burocr\'aticos lo m\'as posible de los desarrolladores. Dejar que los desarrolladores dise\~nen el sistema y tomen las decisiones que consideren apropiadas.
Tener a un usuario involucrado en el proyecto. Este ayuda a encontrar errores, realizar pruebas y perfeccionar el dise\~no del sistema.
Tener un conjunto de pruebas de regresi\'on automatizadas para asegurarse de que el software funciona correctamente en cada paso del desarrollo.
Producir funcionalidad entregable desde temprano y seguido (?).
La metodolog\'ia Crystal Clear se puede describir como sigue:
El dise\~nador principal junto con otras 2 a 7 personas trabajan en un espacio peque\~no como una habitaci\'on (o varias habitaciones adyacentes) con muestras visuales de informaci\'on (como pizarrones) sin ninguna distracci\'on y con acceso directo a usuarios clave.
Deben entregar c\'odigo usable y probado aproximadamente cada 3 meses.
La metodolog\'ia Crystal Clear no pretende ser la mejor, sino tan s\'olo ser "suficiente".
\end{document}